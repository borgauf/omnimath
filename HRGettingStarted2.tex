% Created 2022-12-18 Sun 22:13
% Intended LaTeX compiler: pdflatex
\documentclass[american]{article}
\usepackage[utf8]{inputenc}
\usepackage[T1]{fontenc}
\usepackage{graphicx}
\usepackage{longtable}
\usepackage{wrapfig}
\usepackage{rotating}
\usepackage[normalem]{ulem}
\usepackage{amsmath}
\usepackage{amssymb}
\usepackage{capt-of}
\usepackage{hyperref}
\usepackage{tikz}
\usepackage{tikz}
\usepackage{commath}
\usepackage{pgfplots}
\usepackage{sansmath}
\usepackage{mathtools}
\date{}
\title{Haskell Road; Getting Started Part 2}
\hypersetup{
 pdfauthor={},
 pdftitle={Haskell Road; Getting Started Part 2},
 pdfkeywords={},
 pdfsubject={},
 pdfcreator={Emacs 28.2 (Org mode 9.6)}, 
 pdflang={English}}
\begin{document}

\maketitle

\section*{Getting started part 2}
\label{sec:org7bfc1b4}


\section*{Monday at the library part 2}
\label{sec:org71987b3}

\subsection*{Breaktime}
\label{sec:org4fec33f}

The von der Surwitzes pop over to the student center cafe for a break.
They grab a large mineral water, a brand they knew in Germany, and Ute
has packed some \emph{Vollkornbrot} sandwiches of hummus and cucumber. They
sit at a table and pour the water and pass around out the sandwiches.

𝔘𝔱𝔢: All right, so I
emailed the professor about a couple of questions from that first
chapter of \emph{The Haskell Road}, and she replied saying, first, she's
happy we're tackling the material early. And she mentioned some
resources --- a few texts she has on reserve at the library. \\[0pt]
𝔘𝔴𝔢: Sort of like, I'm
not going to give you the answers. I'm going to point you in the right
direction. What books are they? \\[0pt]
[murmurs of acknowledgement] \\[0pt]
𝔘𝔱𝔢: Math. Upper level
college texts. Abstract algebra and number theory. \\[0pt]
𝔘𝔴𝔢: I've heard
computer science has all these higher math concepts, but then you
don't go as far as a math major does. \\[0pt]
[silence, eating and drinking] \\[0pt]
𝔘𝔴𝔢: [continuing] I
guess you're just supposed to learn as much as you can. But like she
said at the open house, a computer scientist is really an \emph{applied}
mathematician. \\[0pt]
[murmurs of agreement] \\[0pt]
𝔘𝔯𝔰𝔲𝔩𝔞: And the math is
the hardest part for incoming CS students, those first four semesters,
ergo, that's what we're emphasizing it in this course. \\[0pt]
[nods of agreement then silence as they eat and drink] \\[0pt]
𝔘𝔯𝔰𝔲𝔩𝔞: [continuing] So
no hand-waving. And she doesn't have a set amount she wants us to get
through. The course is open-ended. I just find that amazing. \\[0pt]
[murmurs of agreement] \\[0pt]
𝔘𝔴𝔢: But I'm sure we'll
need to keep moving and not be laggards about it. \\[0pt]
[murmurs of agreement] \\[0pt]
𝔘𝔯𝔰𝔲𝔩𝔞: A whole year,
the whole school year. Her sabbatical ends next summer, but I'm pretty
sure I'll want to continue. I don't know if I want to be a computer
scientist or software engineer, but learning this can't hurt. \\[0pt]
𝔘𝔱𝔢: I guess you could
say Novalis is sort of an open \emph{Gynmasium}. \\[0pt]
[soft laughter] \\[0pt]
𝔘𝔴𝔢: And what happens
afterward? They definitely want you to just keep going at the U. Which
I wouldn't mind at all. \\[0pt]
[looks about the table] \\[0pt]
𝔘𝔱𝔢: Yes, and lots of
people just drift into a half-and-half situation where there taking
courses over at the U. \\[0pt]
𝔘𝔯𝔰𝔲𝔩𝔞: Well, Father
has tenure now. But I don't know if Mutti can go on working from
here. [shrugging and sighing] Anyway, I guess you two will cross that
bridge before I will. \\[0pt]
𝔘𝔱𝔢: [laughing] Hardly!
You're right there with us in everything we're doing this coming year.

\subsection*{Divided by}
\label{sec:orgce0892d}
Back at the library study room they've checked out the reserved books
and are looking through sections of those that deal with the basic
theory of division.

𝔘𝔱𝔢: [reading from the
Divisibility section of \emph{Proofs, A Long-Form Mathematical
Textbook}\footnote{\emph{\href{https://longformmath.com/proofs-home}{Proofs; A Long-Form Mathematics Textbook}} by Jay Cummings}] All right, so Professor Chandra wants us to
understand divisibility before we go to \emph{greatest common divisor}, and
before we talk about primes. She said, You have to know all of the
implications of ``divided by'' before you can advance. And like it's
saying in here, [reading] you could just say, \emph{\(a\) divides \(b\) if
\(\frac{a}{b}\) is an integer}. \\[0pt]
[Ursula and Uwe read the section from a second copy] \\[0pt]
𝔘𝔱𝔢: [continuing] But
we don't want that definition, we want \emph{this} definition [getting up
and writing on the board]

\begin{align}
\exists \: k \in \mathbb{Z},\; a \neq 0, \;\;a \mid b \;\; \text{if} \;\; b = a \cdot k  
\end{align}

𝔘𝔱𝔢: [continuing] The
symbol \(a \mid b\:\) means \(a\) divides \(b\) for some \(k\) where \(b = a \cdot
k\;\) and \(a\) is not equal to zero. [pausing] Right, all that makes
sense. So basically, this turns the whole question of divisibility
into finding a proper integer value for \(k\:\) to \emph{multiply} with . Now
we have a math-formalist way of seeing divisibility. \\[0pt]
[murmurs of approval] \\[0pt]
𝔘𝔴𝔢: I like how he says
good definitions don't just fall out of the sky. \\[0pt]
[murmurs of agreement] \\[0pt]
𝔘𝔯𝔰𝔲𝔩𝔞: Then the
examples, like \(2 \mid 14\) is true because \(14 = 2 \cdot 7\:\), in other
words we've found a whole number integer, \(k = 7\:\) and we're
happy. \\[0pt]
𝔘𝔱𝔢: Again, we've
turned division into an issue of true-false logic and
multiplication. [writing on the board] So \(7 \mid 23\) doesn't work
because we have no solution for \(7k = 23\;\). \\[0pt]
𝔘𝔴𝔢: And look at that
last one where it's \(a \mid 0\;\). That's true, for a non-zero \(a\)
since we can say \(0 = a \cdot 0\) is always true for any \(a\) as long as \(k
= 0\;\). \\[0pt]
[murmurs of agreement] \\[0pt]
𝔘𝔱𝔢: And like he says
we're not supposed to look at \(2 \mid 14\) and just say it \emph{equals}
\(7\;\). It's not supposed to be seen as a calculation, it's a logic
\emph{expression} that is true or false --- for some value \(k\:\). \\[0pt]
𝔘𝔴𝔢: Right. We're in
the world of logic now, not grade school arithmetic. So everything has
to be reexplained and reworked. \\[0pt]
[murmurs of agreement] \\[0pt]
𝔘𝔱𝔢: Good, now he's
talking about the \emph{transitive} property of divisibility. It is a
\emph{proposition}, which is a type of theorem, and that means it comes
with a proof. [writing on the board] Here it is in the compact math
logic form

\begin{align*}
a, b, c \in \mathbb{Z},\;\; a \mid b \;\land\; b \mid c \implies a \mid c
\end{align*}

𝔘𝔱𝔢: [continuing] And
then he goes on to prove it by saying assume the \emph{if} part, the \(a
\mid b \;\land\; b \mid c\:\) part is true, that means the \emph{then} part, the
\(a \mid c\) part is true. So [writing]

\begin{align*}
b &= a \cdot s \\[.4em]
c &= b \cdot t
\end{align*}

𝔘𝔱𝔢: [continuing] for
some integers \(s\) and \(t\;\). And now [writing]

\begin{align*}
c &= b \cdot t \\[.4em]
&= (a \cdot s) \cdot t   \\[.4em]
&= a \cdot (s \cdot t) \quad\quad \ldots \; \text{associativity}
\end{align*}

𝔘𝔱𝔢: [continuing] So
since we have the form \(c = a \cdot (s \cdot t)\) where we assumed \(s\) and \(t\)
are integers, and that's the basic form of divisibility, so yes, \(a
\mid c\) since we've shown \(c = a \cdot k\) where \(k = (s \cdot t)\:\). \\[0pt]
𝔘𝔯𝔰𝔲𝔩𝔞: Good. Let's
switch over to this other book [she picks up a Springer Verlag
book\footnote{\emph{\href{https://www.google.com/books/edition/The\_Whole\_Truth\_About\_Whole\_Numbers/ahUGswEACAAJ?hl=en}{The Whole Truth About Whole Numbers}} by Sylvia Forman and
Agnes M. Rash;} and pages through it] Ah, in this book there's a section
called \emph{Divisors and the Greatest Common Divisor}. [paging to section,
reading] Oh, here's one, \emph{Determine whether true or false} [writing on
the board]

\begin{align*}
2 \mid (6n + 4)
\end{align*}

𝔘𝔴𝔢: Interesting. So
writing it in the divisibility way [gets up and writes on the board]

\begin{align*}
(6n + 4) = 2k
\end{align*}

𝔘𝔴𝔢: So before we freak
out and get lost, let's just notice that [writing]

\begin{align}
2(3n + 4) &= 2k \\[.4em]
3n + 4 &= k
\end{align}

𝔘𝔴𝔢: [continuing] I'd
say we don't need to go any further with this. \(2 \mid (6n + 4)\) is
true --- which means it's got solutions --- because \(2\) will go into
\((6n + 4)\) for whatever \(n\) wants to be. \\[0pt]
𝔘𝔱𝔢: And this whole
formal divisibility thing helps because if you just saw this one day
[writing on the board]

\begin{align}
\frac{(6n + 4)}{2} = 3n + 2 
\end{align}

𝔘𝔱𝔢: [continuing]
you've now got a second way to see the idea that the equation is true
for any \(n\), that it's dependent on \(n\;\). \\[0pt]
𝔘𝔯𝔰𝔲𝔩𝔞: [looking
ironically] Thanks, Uwe, Ute, for keeping it real. \\[0pt]
[laughter] \\[0pt]
𝔘𝔱𝔢: [reading text] All
right, we have this example [writing on board]

\begin{align*}
0 \mid 11
\end{align*}

𝔘𝔱𝔢: [continuing] which
is false because there can't be any \(k\) where \(k \cdot 0\) equals
\(11\;\). Agreed? \\[0pt]
[nods of agreement] \\[0pt]
𝔘𝔱𝔢: [continuing] All
right, how about this?

\begin{quote}
Prove that if \(\,a \mid b\) then \(-\, a \mid b\)
\end{quote}

𝔘𝔯𝔰𝔲𝔩𝔞: Let's just
logic it out [getting up and writing on the board]

\begin{align*}
b & = a \cdot k \\[.4em]
b &= (-a) \cdot (-k) \\[.4em]
b &= - (a) \cdot (k) \\[.4em]
b &= - a \cdot k 
\end{align*}

then

\begin{align*}
- a \mid b  \quad \text{for some}\; k \in \mathbb{Z}
\end{align*}

𝔘𝔯𝔰𝔲𝔩𝔞: [continuing] So
\(k\) by virtue of being an integer, which can be either positive or
negative, we've derived \(-\, a \mid b\) from \(a \mid b\;\). \\[0pt]
[silence while the others study the board] \\[0pt]
𝔘𝔴𝔢: Hold it. I'm not
sure we've got the spirit of this, quite. \\[0pt]
𝔘𝔯𝔰𝔲𝔩𝔞: How so? \\[0pt]
𝔘𝔴𝔢: [going to the
board] We need to make sure we understand this as [writing] \((-a) \mid
b\;\) and not as \(-(a \mid b)\:\), right? \\[0pt]
[murmurs of agreement] \\[0pt]
𝔘𝔴𝔢: So that means
we've got [writing] \(b = (-a)(-k)\) as the only possible solution to
keep that \(b\) positive. And I don't think you meant to factor out
\(-1\:\) like you did. So \(k\) must be negative to go with the \(-a\:\),
which then gives positive \(b\;\). That's what is meant, I think. Yes,
\(k\) being an integer allows this. But again, we're dealing with a
multiplicative relationship, we're not doing division. And I'm sure
we'll find out why this is so important in a while. \\[0pt]
𝔘𝔯𝔰𝔲𝔩𝔞: Oh, I think
that was in here. [pulling a large-format book from her messenger
bag\footnote{\emph{\href{http://illustratedtheoryofnumbers.com/}{An Illustrated Theory of Numbers}} by Martin H. Weissman.} and pages to tabbed page]. Right, and he shows that \(0 \mid
0\:\), that zero divides zero, is true --- because [writing on board]
\(0 = 0 \cdot k\:\), meaning \(k\) can be anything and the expression remains
true. [reading further] And he's calling \(k\) the \emph{accessory
number}. [reading further] So his wording is the integers \(x\) that
satisfy \(7 \mid x\) are \(x = 7 \cdot k\) --- and that will be the arithmetic
progression of the multiples of \(7\:\). They're evenly
spaced. Good. And there's this [going to the board and writing] \\[0pt]

\begin{quote}
Plot the integers \(x\) which satisfy \(5 \mid (x - 2)\)
\end{quote}

𝔘𝔱𝔢: [going to the
board and writing] So if that's to be true then we've got \(x - 2 =
5k\:\), and that means for the multiples of \(5\:\), the \emph{set} of
integers \(x\) must keep \(x - 2\) multiples of \(5\) also. So for example

\begin{align*}
-3 - 2 &= 5 \cdot -1 \\[.4em]
2 - 2 &= 5 \cdot 0 \\[.4em]
7 - 2 &= 5 \cdot 1 \\[.4em]
12 - 2 &= 5 \cdot 2 \\[.4em]
\ldots
\end{align*}

𝔘𝔱𝔢: [continuing] And
the so-called \emph{geometric} view of this set of \$x\$s would be a number
line with points [writing on the board]


\begin{center}
\includesvg[width=.9\linewidth]{images/testsine1}
\end{center}



\begin{center}
\includesvg[width=.9\linewidth]{tikztest3}
\end{center}

\begin{center}
\includesvg[width=.9\linewidth]{tikztest2}
\end{center}


\begin{center}
\includesvg[width=.9\linewidth]{tikztest4}
\end{center}


\begin{center}
\includesvg[width=.9\linewidth]{tikztest5a}
\end{center}


\begin{center}
\includesvg[width=.9\linewidth]{tikztest5b}
\end{center}


\begin{center}
\includesvg[width=.9\linewidth]{tikztest5c}
\end{center}

𝔘𝔴𝔢: Good. gold
standard for figuring out lowest common denominator. \\[0pt]
𝔘𝔯𝔰𝔲𝔩𝔞: I'd say so, but
now we need to see how Haskell does it internally, and how \emph{The
Haskell Road\ldots{}} does it and stop being amateurs. \\[0pt]
[laughter] \\[0pt]

𝔘𝔴𝔢: I feel like you
and the professor are like very strong bakers kneading and kneading
and kneading my brain [demonstrates with imaginary brain-dough] \\[0pt]
[laughter] \\[0pt]
𝔘𝔴𝔢: No, this had
really worked out, you, Ursula, racing ahead with the Haskell. And I
going ahead with the set theory, and you, Ute, going on ahead with the
math logic. I mean, we're definitely making progress. It's just that
we have so much to learn! \\[0pt]
[affirmations]
\end{document}